\section{Introduction}\label{sec:intro}
Dans cette section vous trouverez les motivations
principales derrière la rédaction de ce guide.

\subsection{Pourquoi Latex?}\label{subsec:why_latex}

Dans le monde académique et scientifique où le partage de connaissances est primordial,
la rédaction de documents clairs, précis et bien formatés est essentielle.
LaTeX est un système de composition de documents qui répond à ces besoins en offrant:
\begin{itemize}
    \item Une mise en forme professionnelle et cohérente.
    \item Un support avancé pour les mathématiques et les symboles scientifiques.
    \item Une gestion efficace des références bibliographiques et des citations.
    \item La possibilité de créer des documents complexes tels que des thèses, des articles scientifiques, et des présentations.
    \item Une large communauté et de nombreux packages pour étendre ses fonctionnalités.
\end{itemize}

Un utilisateur avancé de LaTeX peut produire des documents d'une qualité 
incomparable à celle des logiciels de traitement de texte classiques comme
Word, LibreOffice ou Google Docs.


\subsection{Ce que vous trouverez dans ce guide}\label{subsec:what_you_will_find}

Ce guide est conçu pour vous aider à naviguer dans l'univers de LaTeX,
il vous offre l'intégralité des outils nécessaires pour rédiger vos documents scientifiques.

Il est rempli de conseils pratiques, d'exemples concrets et de pièges à éviter.

Pour faire une recherche spécifique il suffit de consulter la table des matières.

\subsection{Public visé}\label{subsec:target_audience}

Ce guide s'adresse principalement aux personnes qui utilisent régulièrement des 
mathématiques et aimeraient pouvoir les rédiger de manière professionnelle.
Il est particulièrement utile pour les étudiants, les chercheurs et les enseignants en mathématiques.

Si vous n'êtes pas familier avec LaTeX, ne vous inquiétez pas, ce guide est conçu pour tous les niveaux.

\subsection{Comment utiliser ce guide}\label{subsec:how_to_use_this_guide}

Ce guide est structuré de manière à vous permettre de le consulter à la carte.
Vous pouvez choisir de lire les sections qui vous intéressent le plus,
ou de suivre un parcours complet pour maîtriser LaTeX.
Chaque section est autonome et contient des exemples pratiques que vous pouvez tester directement.