\subsection*{Pourquoi Latex?}\label{subsec:why_latex}
\addcontentsline{toc}{subsection}{\nameref{subsec:why_latex}}

Dans le monde académique et scientifique, la qualité typographique est indissociable de la rigueur intellectuelle. 
Contrairement aux traitements de texte traditionnels (Microsoft Word, Google Docs), 
LaTeX offre un contrôle précis sur la mise en page, des équations parfaitement lisibles, 
et une gestion automatisée des références bibliographiques.

Créé par Leslie Lamport dans les années 1980, LaTeX est devenu \emph{la} norme pour:

\begin{itemize}
    \item Les publications en mathématiques, physique, informatique, et sciences de l’ingénieur.
    \item Les thèses, articles, et prépublications (arXiv).
    \item Les présentations (via Beamer) et posters scientifiques.
\end{itemize}

\subsection*{Ce que vous trouverez dans ce guide}\label{subsec:what_you_will_find}
\addcontentsline{toc}{subsection}{\nameref{subsec:what_you_will_find}}

Ce guide est conçu pour vous aider à naviguer dans l'univers de LaTeX,
il vous offre l'intégralité des outils nécessaires pour rédiger vos documents scientifiques.

Il sera rempli de conseils pratiques, d'exemples concrets et de pièges à éviter.

\subsection*{Public visé}\label{subsec:target_audience}
\addcontentsline{toc}{subsection}{\nameref{subsec:target_audience}}

Ce guide s'adresse principalement aux étudiants et chercheurs en mathématiques,
mais il peut également être utile à toute personne souhaitant améliorer ses compétences en rédaction scientifique.

\subsection*{Comment utiliser ce guide}\label{subsec:how_to_use_this_guide}
\addcontentsline{toc}{subsection}{\nameref{subsec:how_to_use_this_guide}}

Ce guide est structuré de manière à vous permettre de le consulter à la carte.
Vous pouvez choisir de lire les sections qui vous intéressent le plus,
ou de suivre un parcours complet pour maîtriser LaTeX.
Chaque section est autonome et contient des exemples pratiques que vous pouvez tester directement.