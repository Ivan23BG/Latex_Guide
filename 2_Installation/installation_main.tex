Dans cette section nous allons voir comment installer LaTeX sur votre machine.
Nous allons également voir comment installer un éditeur de texte adapté pour LaTeX.

\subsection{LaTeX en ligne}\label{subsec:latex_online}

Il existe plusieurs éditeurs en ligne qui vous permettent de rédiger vos documents LaTeX sans avoir à installer quoi que ce soit sur votre machine.
Ces éditeurs sont particulièrement utiles si vous souhaitez partager votre document avec d'autres personnes ou si vous travaillez sur plusieurs machines.

\begin{itemize}
    \item \textbf{Overleaf}: C'est l'éditeur en ligne le plus populaire. 
    Il offre une interface conviviale et de nombreuses fonctionnalités, y compris la collaboration en temps réel.
    \item \textbf{ShareLaTeX}: Un autre éditeur en ligne qui offre des fonctionnalités similaires à Overleaf.
    \item \textbf{Authorea}: Un éditeur en ligne qui se concentre sur la rédaction scientifique et la collaboration. 
    Il prend en charge LaTeX ainsi que d'autres formats de rédaction.
\end{itemize}

Ces éditeurs en ligne sont gratuits, mais ils offrent également des options payantes pour des fonctionnalités avancées.

Personnellement je n'utilise pas d'éditeur en ligne mais c'est avec ça que j'ai commencé et je
trouve important de les mentionner.

\subsection{Installation de LaTeX}\label{subsec:latex_installation}

Pour installer LaTeX sur votre machine, vous devez télécharger et installer une distribution LaTeX.
Il existe plusieurs distributions LaTeX disponibles selon votre système d'exploitation:

\begin{itemize}
    \item \textbf{TeX Live}: C'est la distribution LaTeX la plus populaire et elle est disponible pour Windows, macOS et Linux,
    \item \textbf{MiKTeX}: Une autre distribution LaTeX populaire, principalement pour Windows,
    \item \textbf{MacTeX}: Une distribution LaTeX spécialement conçue pour macOS.
\end{itemize}

J'utilise personnellement \textbf{MiKTeX} sur Windows.

\subsection{Installation de l'éditeur de texte}\label{subsec:editor_installation}

Il existe de nombreux éditeurs de texte que vous pouvez utiliser pour rédiger vos documents LaTeX.
Certains sont spécialement conçus pour LaTeX, tandis que d'autres sont des éditeurs de texte généraux.
Je vous conseille d'utiliser un éditeur de texte avec lequel vous êtes à l'aise. En voici quelques-uns:

\begin{itemize}
    \item \textbf{TeXworks}: Un éditeur de texte simple et léger, spécialement conçu pour LaTeX.
    \item \textbf{TeXstudio}: Un éditeur de texte plus avancé avec de nombreuses fonctionnalités, 
    y compris la complétion automatique, la gestion des références bibliographiques et l'utilisation de macros.
    \item \textbf{VSCode}: Un éditeur de texte polyvalent qui prend en charge LaTeX via des extensions.
\end{itemize}

J'ai commencé par Overleaf, puis j'ai utilisé TeXstudio pendant un certain temps avant de passer à VSCode.

\subsection{Tester son installation}\label{subsec:test_installation}

Pour tester votre installation, vous pouvez créer un document LaTeX simple et le compiler:

\begin{enumerate}
    \item Créez un nouveau fichier \texttt{test.tex} dans votre éditeur de texte,
    \item Copiez le code suivant dans le fichier: %\lstinputlisting{resources/test.tex}
\begin{lstlisting}[style=latexstyle]
\documentclass{article}
\usepackage{amsmath}
\begin{document}
Hello, \LaTeX{}!
\end{document}
\end{lstlisting}
    \item Compilez le fichier en ligne de commande par exemple avec \texttt{pdflatex test.tex}.
\end{enumerate}

Si tout fonctionne correctement, vous devriez obtenir un fichier PDF nommé \texttt{test.pdf} contenant le texte \og{}Hello, LaTeX!\fg{} et une équation intégrale.

Si vous obtenez également plein d'autres fichiers, ne paniquez pas, c'est normal.
LaTeX génère plusieurs fichiers auxiliaires lors de la compilation, on verra comment les utiliser plus tard.

\subsection{Options avancées}\label{subsec:advanced_options}

Si vous souhaitez aller plus loin, il existe plusieurs options avancées pour personnaliser votre installation LaTeX.

\subsubsection{Gestion des packages}\label{subsubsec:package_management}

La première étape importante pour écrire des documents avancés est de gérer ses packages.

LaTeX utilise des packages pour ajouter des fonctionnalités supplémentaires à votre document.
Vous pouvez installer des packages supplémentaires en utilisant le gestionnaire de packages de votre distribution LaTeX
ou en les téléchargeant manuellement.

\subsubsection{Compilateurs}\label{subsubsec:compilers}

LaTeX prend en charge plusieurs compilateurs, chacun ayant ses propres avantages et inconvénients.
Les plus courants sont:

\begin{itemize}
    \item \textbf{pdfLaTeX}: Le compilateur par défaut, qui génère des fichiers PDF,
    \item \textbf{XeLaTeX}: Un compilateur qui prend en charge les polices OpenType et TrueType,
    ainsi que les langues non latines,
    \item \textbf{LuaLaTeX}: Un compilateur qui utilise le moteur Lua pour la programmation et la personnalisation avancée,
    \item \textbf{Latexmk}: Un script Perl qui automatise le processus de compilation de documents LaTeX.
\end{itemize}

Pour des documents simples, pdfLaTeX est généralement suffisant.
Pour des projets plus avancés il est souvent préférable de passer plutôt par Latexmk.