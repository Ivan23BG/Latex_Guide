\documentclass{article}  % Classe du document (article, report, book, etc.)
\usepackage[utf8]{inputenc}  % Encodage des caracteres (obligatoire pour les accents)
\usepackage[T1]{fontenc}  % Meilleure gestion des fontes (cesures, accents copiables)

% Packages utiles (ajoutez selon vos besoins)
\usepackage{amsmath}  % Pour les mathematiques
\usepackage{graphicx}  % Pour les images

% Debut du document
\begin{document}

\title{Titre de l'Article}
\author{Votre Nom}
\date{\today}  % Date automatique
\maketitle  % Affiche le titre

\section{Introduction}
Voici un paragraphe. LaTeX gere automatiquement les marges, l'interligne et la justification.

\subsection{Une sous-section}
Les sous-sections sont numerotees automatiquement.

\section{Mathematiques}
Les equations en ligne s'ecrivent avec \(E=mc^2\), et les equations hors ligne avec :
\[
\int_0^1 x^2 \, \mathrm{d}x = \frac{1}{3}.
\]

\end{document}