Pour ceux qui veulent du concret sans se perdre dans les details,
la section suivante présente la structure de base d'un document LaTeX.

\subsection{Le squelette}\label{subsec:latex_skeleton}
\begin{listing}[H]
\inputminted[frame=single,framesep=4pt,breaklines]{latex}{./3_Structure/latex_skeleton.tex}
\caption{Squelette d'un document LaTeX}
\label{lst:latex_skeleton}
\end{listing}

\subsection{Les différentes parties d'un document}\label{subsec:latex_parts}

Un document LaTeX s'écrit en plusieurs parties:

\begin{itemize}
    \item \textbf{Préambule}: C'est la première partie du document, où vous définissez le type de document, les packages à utiliser et d'autres paramètres globaux.

    Dans le code~\ref{lst:latex_skeleton}, le préambule est constitué des lignes 1 à 7.
    On commence par définir le type de document, ici un article (ce sera souvent le cas)
    et puis on charge les packages nécessaires.

    \item \textbf{Corps du document}: C'est la partie principale du document, 
    où vous écrivez le contenu de votre document, y compris les sections, les sous-sections, 
    les figures, les tableaux, etc.

    Dans le code~\ref{lst:latex_skeleton}, le corps du document est constitué des lignes 10 à 29.
\end{itemize}

Le préambule commence souvent avec les mêmes packages mais 
n'hésitez pas à ajouter des packages plus tard si vous en avez besoin.

\subsection{Les environnements}\label{subsec:latex_environments}

Un environnement est une partie du document qui a un comportement particulier.
Il est délimité par des commandes \texttt{\textbackslash{}begin\{\ldots\}} et \texttt{\textbackslash{}end\{\ldots\}}.

Il existe de nombreux environnements prédéfinis dans LaTeX, chacun ayant une fonction spécifique.
Il est même possible de créer ses propres environnements.

Les plus utiles sont:
\begin{itemize}
    \item \texttt{document}: C'est l'environnement principal qui contient le corps du document.
    \item \texttt{itemize}: C'est un environnement pour créer des listes à puces.
    \item \texttt{enumerate}: C'est un environnement pour créer des listes numérotées.
    \item \texttt{figure}: C'est un environnement pour insérer des figures.
    \item \texttt{equation}: C'est un environnement pour écrire des équations mathématiques.
    \item \texttt{table}: C'est un environnement pour créer des tableaux.
\end{itemize}

Mais il en existe beaucoup d'autres et vous verrez plus tard comment les utiliser.

\subsection{Les commandes}\label{subsec:latex_commands}
Une commande est une instruction qui dit à LaTeX de faire quelque chose.
Elle commence généralement par un backslash (\texttt{\textbackslash}) et peut avoir des arguments entre accolades \texttt{\{\}} ou entre crochets \texttt{[ ]}.
Les arguments entre accolades sont obligatoires, tandis que ceux entre crochets sont optionnels.

Par exemple, la commande \texttt{\textbackslash{}section\{Titre\}} crée une section avec le titre \og{}Titre\fg{}.
La commande \texttt{\textbackslash{}textbf\{Texte\}} met le mot \og{}Texte\fg{} en gras.
Il existe de nombreuses commandes prédéfinies dans LaTeX. Vous pouvez également créer vos propres commandes.

D'ailleurs, les commandes custom sont souvent très utiles pour éviter de répéter du code.
Vous pouvez les définir dans le préambule de votre document.

\subsection{Les éléments de mise en forme}\label{subsec:latex_formatting}
LaTeX propose de nombreux éléments de mise en forme pour personnaliser l'apparence de votre document.
Voici quelques-uns des plus courants:
\begin{itemize}
    \item \texttt{\textbackslash{}textbf\{\}}: Met le texte en gras.
    \item \texttt{\textbackslash{}textit\{\}}: Met le texte en italique.
    \item \texttt{\textbackslash{}underline\{\}}: Souligne le texte.
    \item \texttt{\textbackslash{}textcolor\{couleur\}\{\}}: Change la couleur du texte.
\end{itemize}

Pour ce qui est des éléments de mise en forme du document, il y a:

\begin{itemize}
    \item \texttt{\textbackslash{}title\{\}}: Définit le titre du document.
    \item \texttt{\textbackslash{}author\{\}}: Définit l'auteur du document.
    \item \texttt{\textbackslash{}date\{\}}: Définit la date du document.
    \item \texttt{\textbackslash{}maketitle}: Génère la page de titre du document.
    \item \texttt{\textbackslash{}tableofcontents}: Génère la table des matières du document.
    \item \texttt{\textbackslash{}section\{\}}: Crée une nouvelle section dans le document.
    \item \texttt{\textbackslash{}subsection\{\}}: Crée une nouvelle sous-section dans le document.
\end{itemize}

\subsection{Ce que vous devez retenir}\label{subsec:latex_what_to_remember}

Si vous avez bien regardé tous les exemples ci-dessus, vous avez dû remarquer que
la plupart des commandes et environnements sont en anglais. Plus fort que ça,
si vous avez besoin d'un certain type d'élément ou d'effectuer une certaine action,
il est fort probable que cette commande existe déjà et qu'il suffit de trouver son nom (en anglais).

Il est donc important de bien se familiariser avec la documentation de LaTeX et de ses packages.
Au début vous passerez beaucoup de temps à chercher de nouvelles commandes. Mais avec la pratique votre rédaction sera plus fluide et automatique.