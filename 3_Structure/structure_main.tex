\section{Structure d'un document}\label{sec:structure}
Pour ceux qui veulent du concret sans se perdre dans les details,
la section suivante présente la structure de base d'un document LaTeX.

\subsection{Le squelette}\label{subsec:latex_skeleton}
Un document LaTeX, contrairement à un document Word, est écrit en code.
Il est donc important de comprendre les différentes parties qui le composent.

\subsubsection{La classe de document}
La première ligne d'un document LaTeX est la déclaration de la classe de document.
C'est une commande qui indique à LaTeX quel type de document vous allez rédiger.
La classe de document détermine la mise en page, la taille de la police, les marges, etc.
Elle ressemble à ceci:
\begin{minted}[frame=single, breaklines]{latex}
\documentclass{article}
\end{minted}
La classe \texttt{article} est la plus courante pour les articles scientifiques.
Il existe d'autres classes comme \texttt{report} pour les rapports, \texttt{book} pour les livres, etc.

\subsubsection{Les packages}
Les packages sont des extensions qui ajoutent des fonctionnalités à LaTeX.
C'est ici que vous allez charger les packages dont vous avez besoin pour votre document.
Vous utiliserez souvent beaucoup de packages, ils ressemblent à ceci:
\begin{minted}[frame=single, breaklines]{latex}
\usepackage[utf8]{inputenc} % Utile pour les accents
\usepackage[T1]{fontenc} % Utile pour les caractères spéciaux
\usepackage{amsmath} % Utile pour les mathématiques
\usepackage{graphicx} % Utile pour les figures
\end{minted}
Vous pouvez charger autant de packages que vous le souhaitez.

Remarquez que les deux premiers packages ne sont pas importés comme les autres.
Ils sont chargés avec des options entre crochets, c'est-à-dire qu'ils acceptent des paramètres.
C'est le cas pour beaucoup de packages mais aussi pour certaines commandes.

\subsubsection{Le corps du document}
Le corps du document contient le contenu de votre document.
Il ressemble à ceci:
\begin{minted}[frame=single, breaklines]{latex}
\begin{document}
% --- Les lignes suivantes sont optionnelles --- %
\title{Titre de votre document}
\author{Votre nom}
\date{\today} % Date du jour
\maketitle % Génère la page de titre
\tableofcontents % Génère la table des matières
% --- Fin des lignes optionnelles --- %

\section{Exemple de section}
Voici un exemple de section dans votre document et son contenu.

\subsection{Exemple de sous-section}
Voici un exemple de sous-section dans votre document et son contenu.

\end{document}
\end{minted}

Il est délimité par les commandes \mintinline{latex}{\begin{document}} et \mintinline{latex}{\end{document}}.
On peut y mettre du texte, des sections, des sous-sections, des figures, des tableaux, etc.
Il est important de noter que tout ce qui est écrit en dehors de cet environnement ne sera pas 
affiché dans le document final.

\subsubsection{Le code complet}
Voici un exemple complet de squelette d'un document LaTeX.

\begin{listing}[H]
\inputminted[frame=single, breaklines]{latex}{./3_Structure/latex_skeleton.tex}
\caption{Squelette d'un document LaTeX}
\label{lst:latex_skeleton}
\end{listing}


\subsection{Les environnements}\label{subsec:latex_environments}

Un environnement est une partie du document qui a un comportement particulier.
Il est délimité par des commandes \mintinline{latex}{\begin{example}} et \mintinline{latex}{\end{example}}.

Il existe de nombreux environnements prédéfinis dans LaTeX, chacun ayant une fonction spécifique.
Il est même possible de créer ses propres environnements. Les plus utiles sont:
\begin{itemize}
    \item \texttt{document}: C'est l'environnement principal qui contient le corps du document.
    \item \texttt{itemize}: C'est un environnement pour créer des listes à puces.
    \item \texttt{enumerate}: C'est un environnement pour créer des listes numérotées.
    \item \texttt{figure}: C'est un environnement pour insérer des figures.
    \item \texttt{equation}: C'est un environnement pour écrire des équations mathématiques.
    \item \texttt{table}: C'est un environnement pour créer des tableaux.
\end{itemize}

Mais il en existe beaucoup d'autres et vous verrez plus tard comment les utiliser.

\subsection{Les commandes}\label{subsec:latex_commands}
Une commande est une instruction qui dit à LaTeX de faire quelque chose.
Elle commence généralement par un backslash \texttt{\textbackslash} suivi du nom de la commande.
Les commandes peuvent prendre des arguments, qui sont les informations que vous voulez passer à la commande.

Les arguments entre accolades sont obligatoires, tandis que ceux entre crochets sont optionnels.

Notez que les commandes custom sont souvent très utiles pour éviter de répéter du code.
Vous pouvez les définir dans le préambule de votre document 
de la manière suivante:
\begin{minted}[frame=single, breaklines]{latex}
\newcommand{\maCommande}[nbargs]{effet de la commande}
\end{minted}
Où \texttt{maCommande} est le nom de la commande que vous voulez créer,
\texttt{nbargs} est le nombre d'arguments que la commande prend (0, 1, 2, etc.),
et \texttt{effet de la commande} est le code que vous voulez exécuter lorsque la commande est appelée.

En voici un exemple:
\begin{minted}[frame=single, breaklines]{latex}
\newcommand{\maCommande}[2]{%
    \textbf{#1} % Met le premier argument en gras
    \textit{#2} % Met le deuxième argument en italique
}
% On l'appelle comme ceci:
\maCommande{Texte en gras}{Texte en italique}
\end{minted}

\subsection{Les éléments de mise en forme}\label{subsec:latex_formatting}
LaTeX propose de nombreux éléments de mise en forme pour personnaliser l'apparence de votre document.
En voici quelques-uns:
\begin{itemize}
    \item \mintinline{latex}{\textbf{exemple}}: Met le texte en gras.
    \item \mintinline{latex}{\underline{exemple}}: Souligne le texte.
    \item \mintinline{latex}{\color{red}{exemple}}: Change la couleur du texte (nécessite le package \texttt{xcolor}).
\end{itemize}

Pour ce qui est des éléments de mise en forme du document, comme
les titres, les sections, les sous-sections, etc.\ elles
sont dans la figure~\ref{lst:latex_skeleton} du squelette d'un document LaTeX.

\subsection{Ce que vous devez retenir}\label{subsec:latex_what_to_remember}

Si vous avez bien regardé tous les exemples ci-dessus, vous avez dû remarquer que
la plupart des commandes et environnements sont en anglais. Plus fort que ça,
si vous avez besoin d'un certain type d'élément ou d'effectuer une certaine action,
il est fort probable que cette commande existe déjà et qu'il suffit de trouver son nom (en anglais).

Il est donc important de bien se familiariser avec la documentation de LaTeX et de ses packages.
Au début vous passerez beaucoup de temps à chercher de nouvelles commandes. 
Mais avec la pratique votre rédaction sera plus fluide et automatique.